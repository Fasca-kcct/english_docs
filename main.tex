\documentclass[xcolor=dvipsnames,unicode,14pt]{beamer}% 'unicode'が必要
\usepackage{luatexja}% 日本語したい
\usepackage[ipaex]{luatexja-preset}% IPAexフォントしたい
\renewcommand{\kanjifamilydefault}{\gtdefault}% 既定をゴシック体に
\usepackage{times}

% あとは欧文の場合と同じ
\usetheme{Darmstadt}

\title{分詞構文(時,理由)}


\begin{document}


\frame{\maketitle}

\begin{frame}
  \tableofcontents
\end{frame}
\section{分詞構文の基礎}
\begin{frame}
  \centering
  \LARGE 分詞について

\end{frame}
\begin{frame}
  \frametitle{分詞の復習}
  \begin{block}{分詞}
    動詞が形を変えて形容詞の働きをするようになったもので,
    現在分詞と過去分詞の2種類がある。
    主に形容詞句を作って名詞を修飾する。
  \end{block}
  \begin{block}{訳し方}
    現在分詞の場合,「\textasciitilde している」と訳す。
    過去分詞の場合,「\textasciitilde された」と訳す。
  \end{block}
\end{frame}

\begin{frame}
  \frametitle{語・句・節}
  語・句・節とは
  \pause
  \begin{description}
    \item[語:] 単語1語1語のこと
    \item[句:] 2つ以上の語が集まりひとつの意味のかたまりを成し,1つの品詞のような働きをするもの
    \item[節:] 意味のひとかたまりの中に主語と述語動詞の関係(S+V)があって、その部分が文として独立していないもの
  \end{description}
  \pause

\end{frame}
\begin{frame}
  \frametitle{分詞構文とは}

  \begin{block}{分詞構文}
    分詞句が文を修飾する副詞の働きをするもの

    二つの文を接続詞無しで繋げる

    二つの文の主語は\textcolor{Red}{\underline{\textcolor{black}{一致している必要がある}}}
  \end{block} 
  \textcolor{NavyBlue}{\underline{Written \textcolor{black}{in plain English}}} ,\\ \qquad \qquad \qquad\qquad\qquad this book is easy to read. 

  \underline{わかりやすい英語で書かれているので},\\ この本は読みやすい。

  \vskip.5\baselineskip
  →分詞が導く句が理由を表している

\end{frame}

\begin{frame}
  \frametitle{分詞構文の否定表現}

  分詞構文で分詞が作る句を否定表現にするには分詞の前にnotをつける。

  \vskip.5\baselineskip

  \textcolor{NavyBlue}{Not having a car} , he has to walk to his office.

  \textcolor{NavyBlue}{車を持っていないので},彼はオフィスまで歩かなければならない。

  \vskip.5\baselineskip
  これは次のように書き換えられる。

  \textcolor{NavyBlue}{Because he doesn't have a car}, he has to walk to his office.

\end{frame}



\section{分詞構文(理由)}
\begin{frame}
  \centering
  \LARGE 分詞構文(理由)

\end{frame}
\begin{frame}
  \frametitle{理由を表す分詞構文}

  \textcolor{NavyBlue}{\underline{Having \textcolor{black}{no money}}}, I did'nt see the movie.

  \underline{お金がなかったので},私はその映画を観れなかった。

  \vskip.5\baselineskip

  →分詞が作る句がその後の文の理由となっている。

  \vskip.5\baselineskip

  訳し方は二つの文を「ので」で繋げる。
\end{frame}

\begin{frame}
  \frametitle{分詞構文(理由)の書き換え}

  理由を表す分詞構文は接続詞"because"を用いて書き換えられる。

  \vskip.5\baselineskip

  →分詞構文は接続詞を省略していると考えればよい

  \vskip.5\baselineskip

  \textcolor{NavyBlue}{\underline{Having \textcolor{black}{no money}}}, I did'nt see the movie.

  → I did'nt see the movie \textcolor{NavyBlue}{because}
   I \textcolor{Red}{\underline{had \textcolor{black}{no money.}}}

   \vskip.5\baselineskip
  もとの分詞の動詞の時制に注意!!

\end{frame}


\section{分詞構文(時)}
\begin{frame}
  \centering
  \LARGE 分詞構文(時)

\end{frame}
\begin{frame}
  \frametitle{時を表す分詞構文}

  He broke his leg \textcolor{NavyBlue}{\underline{playing 
  \textcolor{black}{soccer}}}.

  彼は\textcolor{NavyBlue}{サッカーをしていたとき}に足を折った。

  \vskip.5\baselineskip

  →主文が起こった時に何があったかを分詞が作る句が説明している。

  \vskip.5\baselineskip

  訳し方は二つの文を「とき」や「間」で繋げる。

  \vskip.5\baselineskip

  →どちらを使うかは文脈で判断する。

\end{frame}
\begin{frame}
  \frametitle{時を表す分詞構文の書き換え}

  時を表す分詞構文は接続詞"when"もしくは"while"を用いて書き換えられる。

  \vskip.5\baselineskip

  He broke his leg \textcolor{NavyBlue}{\underline{playing 
  \textcolor{black}{soccer}}}.

  →He broke his leg \textcolor{NavyBlue}{\underline{while he was playing 
  \textcolor{black}{soccer}}}.

  \vskip.5\baselineskip

  また,接続詞の後ろのS,Vは省略されることもある。

  \vskip.5\baselineskip

  He broke his leg \textcolor{NavyBlue}{\underline{while he was playing 
  \textcolor{black}{soccer}}}.

  →He broke his leg \textcolor{NavyBlue}{\underline{while playing 
  \textcolor{black}{soccer}}}.

\end{frame}

\section{おまけ}
\begin{frame}
  \centering
  \LARGE おまけ

\end{frame}
\begin{frame}
  \frametitle{分詞構文でよく使われるフレーズ1}

  \begin{itemize}
    \item Speaking of \textasciitilde \\ 
          →\textasciitilde と言えば \\
          例文:\textcolor{NavyBlue}{Speaking of movies}, what kind of movies do you like? \\ 
          訳:\textcolor{NavyBlue}{映画と言えば}、どのような映画が好きですか?
    \item Judging from\textasciitilde \\ 
        →\textasciitilde から判断すると \\
        例文:\textcolor{NavyBlue}{Judging from the reviews}, I think I should read this book. \\ 
        訳:\textcolor{NavyBlue}{レビューから判断すると}、私はこの本を読むべきだと思う。
  \end{itemize}

\end{frame}
\begin{frame}
  \frametitle{分詞構文でよく使われるフレーズ2}

  \begin{itemize}
    \item Considering \textasciitilde \\ 
          →\textasciitilde を考えると \\
          例文:\textcolor{NavyBlue}{Considering her age}, I don't think he can climb the mountain. \\ 
          訳:\textcolor{NavyBlue}{彼女の年齢を考えると}、私は彼女がその山に登れるとは思わない。
    \item Taking A into consideration \\ 
        → Aを考慮に入れた場合 \\
        例文:\textcolor{NavyBlue}{Taking everything into consideration}, do you think you should buy a house? \\ 
        訳:\textcolor{NavyBlue}{全てのことを考慮に入れた場合}、家を購入すべきだと思いますか?
  \end{itemize}

\end{frame}
\begin{frame}
  \frametitle{分詞構文でよく使われるフレーズ3}

  \begin{itemize}
    \item Frankly speaking  \\ 
          →率直に言って \\
          \vskip.5\baselineskip
          例文:\textcolor{NavyBlue}{Frankly speaking,} I don't like him.\\ 
          訳:\textcolor{NavyBlue}{率直に言って}、私は彼が好きではない。
    \item Generally speaking\\ 
        → 一般的に言って \\
        \vskip.5\baselineskip
        例文:\textcolor{NavyBlue}{Generally speaking}, men are taller than women. \\ 
        訳:\textcolor{NavyBlue}{一般的に言って}、男性は女性より背が高い。
  \end{itemize}

\end{frame}
\begin{frame}
  \frametitle{分詞構文でよく使われるフレーズ4}

  \begin{itemize}
    \item Strictly speaking  \\ 
          →厳密に言うと \\
          \vskip.5\baselineskip
          例文:\textcolor{NavyBlue}{Strictly speaking}, his opinion is different from mine.\\ 
          訳:\textcolor{NavyBlue}{厳密に言うと}、彼の意見は私のものとは違う。
  \end{itemize}

\end{frame}

% \begin{frame}
%   \frametitle{分詞構文と完了形}

%   \textcolor{NavyBlue}{\underline{Having read \textcolor{black}{the book}}} , she returned it to the library.


% \end{frame}







\end{document}