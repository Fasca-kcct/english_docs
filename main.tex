\documentclass[xcolor=dvipsnames,unicode,14pt]{beamer}% 'unicode'が必要
\usepackage{luatexja}% 日本語したい
\usepackage[ipaex]{luatexja-preset}% IPAexフォントしたい
\renewcommand{\kanjifamilydefault}{\gtdefault}% 既定をゴシック体に
\usepackage{times}

% あとは欧文の場合と同じ
\usetheme{Darmstadt}

\title{分詞構文(時,理由)}


\begin{document}


\frame{\maketitle}

\begin{frame}
  \tableofcontents
\end{frame}
\section{分詞構文の基礎}
\begin{frame}
  \frametitle{分詞の復習}
  \begin{block}{分詞}
    動詞が形を変えて形容詞の働きをするようになったもので,
    現在分詞と過去分詞の2種類がある。
    主に形容詞句を作って名詞を修飾する。
  \end{block}
  \begin{block}{訳し方}
    現在分詞の場合,「\textasciitilde している」と訳す。
    過去分詞の場合,「\textasciitilde された」と訳す。
  \end{block}
\end{frame}

\begin{frame}
  \frametitle{語・句・節}
  語・句・節とは
  \pause
  \begin{description}
    \item[語] 単語1語1語のこと
    \item[句] 2つ以上の語が集まりひとつの意味のかたまりを成し,1つの品詞のような働きをするもの
    \item[節] 意味のひとかたまりの中に主語と述語動詞の関係(S+V)があって、その部分が文として独立していないもの
  \end{description}
  \pause

\end{frame}
\begin{frame}
  \frametitle{分詞構文とは}

  \begin{block}{分詞構文}
    分詞が導く句が文を修飾する副詞の働きをするもの

    二つの文を接続詞無しで繋げる

    二つの文の主語は\textcolor{Red}{\underline{\textcolor{black}{一致している必要がある}}}
  \end{block} 
  \textcolor{NavyBlue}{\underline{Written \textcolor{black}{in plain English}}} , this book is easy to read. 

  わかりやすい英語で書かれているので,この本は読みやすい。

  →分詞が導く句が理由を表している

\end{frame}



\section{分詞構文(理由)}
\begin{frame}
  \frametitle{理由を表す分詞構文}

  \textcolor{NavyBlue}{\underline{Having \textcolor{black}{no money}}}, I did'nt see the movie.

  お金がなかったので,私はその映画を観れなかった。

  →分詞が作る句がその後の文の理由となっている。

  訳し方は二つの文を「ので」で繋げる。
\end{frame}

\begin{frame}
  \frametitle{分詞構文(理由)の書き換え}

  理由を表す分詞構文は接続詞"because"を用いて書き換えられる。

  \textcolor{NavyBlue}{\underline{Having \textcolor{black}{no money}}}, I did'nt see the movie.

  → I did'nt see the movie \textcolor{NavyBlue}{because}
   I \textcolor{Red}{\underline{had \textcolor{black}{no money.}}}

  もとの分詞の動詞の時制に注意!!

\end{frame}
\begin{frame}
  \frametitle{否定表現}

  分詞構文で分詞が作る句を否定表現にするには分詞の前にnotをつける。

  \textcolor{NavyBlue}{Not having a car} , he has to walk to his office.

  車を持っていないので、彼はオフィスまで歩かなければならない。

  これは次のように書き換えられる。

  \textcolor{NavyBlue}{Because he doesn't have a car}, he has to walk to his office.

\end{frame}

\section{分詞構文(時)}
\begin{frame}
  \frametitle{時を表す分詞構文}

  

\end{frame}




\end{document}